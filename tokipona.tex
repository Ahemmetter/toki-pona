\documentclass[10pt,a4paper]{article}
\usepackage[utf8]{inputenc}
\usepackage[english, esperanto, russian]{babel}
\usepackage[left=1.5cm,right=1.5cm,top=1.0cm,bottom=1.5cm]{geometry}
\author{Andreas Hemmetter}
\usepackage{multicol}
\usepackage{textcomp}
\usepackage{amsthm}
\usepackage{framed}
\usepackage{multirow}
\usepackage{bbold}
\usepackage{amssymb}
\usepackage{hyperref}
\usepackage{booktabs}
\usepackage{graphicx}
%\usepackage{braket}
\usepackage{tikz}
\usetikzlibrary{arrows}

\begin{document}
\pagestyle{empty}
\begin{center}
\rule{40mm}{2pt}$~\big\{$\textbf{\textsc{{\large  Toki Pona}}}$\big\}~$\rule{40mm}{2pt}\\
by \href{mailto:a.hemmetter@gmail.com}{Andreas Hemmetter}
\rule{15cm}{0.2pt}
\end{center}

\paragraph{What is Toki Pona?}
Toki Pona is a minimalist language, designed by Canadian linguist Sonja Lang to help simplify one's thoughts. It is based on Taoism and the idea that already extremely simple grammar and vocabulary allows people to communicate necessary information. If you're not into Taoism, think of a group of people stranded on an island who cannot understand each other - simple communication regardless of native language can easily be introduced with just this piece of paper and a few hours of time, which they seem to have a lot of anyway.

\paragraph{Alphabet and Spelling}

Toki Pona uses only sounds that are common to most languages. These are \textbf{\textsc{k}}, \textbf{\textsc{l}}, \textbf{\textsc{m}}, \textbf{\textsc{n}}, \textbf{\textsc{p}}, \textbf{\textsc{s}}, \textbf{\textsc{t}}, \textbf{\textsc{w}} and \textbf{\textsc{j}} (as in \textbf{y}et) as consonants and \textbf{\textsc{a}} (f\textbf{a}ther), \textbf{\textsc{e}} (m\textbf{e}t), \textbf{\textsc{i}} (p\textbf{ee}l), \textbf{\textsc{o}} (m\textbf{o}re) and \textbf{\textsc{u}} (f\textbf{oo}d) as vowels. Possible syllables follow the CV(N) pattern, where N is a nasal sound. Words are never capitalized, except for proper names. Due to its limited phonology, Toki Pona can be written in pretty much any writing system, including Hangul, Arabic, Cyrillic and hieroglyphs.

\paragraph{Vocabulary}

Toki Pona has a vocabulary of 120-ish words and is therefore quite ambiguous. Words don't have a defined function, gender, case, tense, number or even a properly defined meaning. Instead, they convey \textit{concepts}, and their function becomes clear from context. For example \textbf{\textsc{mi moku}} can mean \textit{I eat / I will eat / I ate / I am food} etc.

\paragraph{Sentence Structure}

Sentences are formed much in the same way simple English sentences are made. Several things are, however, to watch out for. Toki Pona requires identifier \textbf{\textsc{li}} to separate subject and verb (except after \textbf{\textsc{mi}} and \textbf{\textsc{sina}}) and \textbf{\textsc{e}} to separate verb and direct object. For example: \textbf{\textsc{ona li pona e ilo}} \textit{She is fixing the tool}. Furthermore, multiple \textbf{\textsc{li}} or \textbf{\textsc{e}} can be used as \textit{and}: \textbf{\textsc{pipi li lukin li moku}} \textit{The bug looks and eats} and \textbf{\textsc{mi moku e kili e telo}} \textit{I eat fruit and water}. There is no \textit{to be}, so \textbf{\textsc{mi pona}} means \textit{I am good}.

\paragraph{Compound Words}

Due to the small vocabulary, more complicated words need to be expressed through compound words by adding adjectives \textit{after} the main concept: \textbf{\textsc{jan}} \textit{person}, \textbf{\textsc{jan utala}} \textit{soldier (fighting person)}, \textbf{\textsc{jan utala nasa}} \textit{stupid soldier}, etc. Any word, including \textbf{\textsc{mute}} \textit{many}, \textbf{\textsc{ni}} \textit{this} and the pronouns can act as adjective/adverb after the noun/verb: \textbf{\textsc{mi utala ike}} \textit{I fight badly}.

\paragraph{Prepositions}

There are no defined prepositions in Toki Pona, but rather words which loosely describe a relation. These are \textbf{\textsc{lon}} \textit{to be in/at something}, \textbf{\textsc{kepeken}} \textit{to use with something}, \textbf{\textsc{tawa}} \textit{to move to somewhere}, \textbf{\textsc{kama}} \textit{to come/cause}, \textbf{\textsc{sama}} \textit{like}, \textbf{\textsc{tan}} \textit{because} and \textbf{\textsc{poka}} \textit{beside}.

\begin{multicols}{2}
\noindent\textbf{\textsc{suno li lon sewi}} \textit{The sun is in the sky}\\
\noindent\textbf{\textsc{mi wile e ni: mi lon tomo}} \textit{I want to be at home}\\

\noindent\textbf{\textsc{mi moku tan ni: mi wile moku}} \textit{I eat because I am hungry}\\

\noindent\textbf{\textsc{mi tawa tomo mi}} \textit{I am goingt to my house}\\
\noindent\textbf{\textsc{mi toki tawa sina}} \textit{I talk to you}\\
\noindent\textbf{\textsc{ni li pona tawa mi}} \textit{That is good for me (I like that)}\\
\noindent\textbf{\textsc{mi tawa e kiwen}} \textit{I am moving the rock}\\

\noindent\textbf{\textsc{ona li kama tawa tomo mi}} \textit{He came to my house}\\
\noindent\textbf{\textsc{mi kama e pakala}} \textit{I caused an accident}\\
\noindent\textbf{\textsc{mi kama jo e telo}} \textit{I'm getting water}\\

\noindent\textbf{\textsc{mi moku kepeken ilo moku}} \textit{I eat with a fork/spoon}\\
\noindent\textbf{\textsc{mi kepeken e poki}} \textit{I use a cup}\\

\noindent\textbf{\textsc{mi moku poka jan pona mi}} \textit{I ate beside my friend}\\

\noindent\textbf{\textsc{jan ni li sama mi}} \textit{That person is like me}\\

\noindent Other nouns which work as prepositions:\\
\noindent\textbf{\textsc{ona li lon sewi mi}} \textit{He is above me}\\
\noindent\textbf{\textsc{pipi li lon anpa me}} \textit{The bug is underneath me}\\
\noindent\textbf{\textsc{moku li lon insa mi}} \textit{Food is in my stomach}\\
\noindent\textbf{\textsc{len li lon poka mi}} \textit{The clothes are at my side}
\end{multicols}

\paragraph{Negation and Questions}

Sentences are negated by placing \textbf{\textsc{ala}} \textit{not} after the verb: \textbf{\textsc{mi wile ala tawa musi}} \textit{I don't want to dance}; it essentially acts as an adjective (like also \textbf{\textsc{ali}} \textit{all}). Yes/No questions are formed by repeating the verb after \textbf{\textsc{ala}}: \textbf{\textsc{sina pona ala pona?}} \textit{Are you okay?}. To answer, repeat the verb with or without \textbf{\textsc{ala}}. To ask for the subject, \textbf{\textsc{seme}} is used: \textbf{\textsc{seme li lon tomo mi?}} \textit{What is in my house?}. It can also be used to ask for the direct object (\textbf{\textsc{sina lukin e seme?}} \textit{What are you watching?}), the person (\textbf{\textsc{jan seme li moku?}} \textit{Who is eating?}), the reason (\textbf{\textsc{sina kama tan seme?}} \textit{Why did you come?}) or for a specific thing (\textbf{\textsc{ma seme li pona tawa sina?}} \textit{Which countries do you like?}). The word \textbf{\textsc{anu}} \textit{either/or} gives a choice between two options: \textbf{\textsc{sina jo e kili anu telo nasa?}} \textit{Do you have fruit, or is it the wine that you have?}.

\paragraph{Details}
\begin{multicols}{2}
\begin{itemize}
\item Gender can be explicitly shown by using the adjectives \textbf{\textsc{meli}} \textit{female} and \textbf{\textsc{mije}} \textit{male}, if necessary.
\item Country names are always adjectives and follow the syllable formation rules of Toki Pona: \textbf{\textsc{ma Kanata}} \textit{(the country of) Canada}. Same goes for languages (\textbf{\textsc{toki}}), nationalities (\textbf{\textsc{jan}}) and names (\textbf{\textsc{jan}}).
\item The imperative is formed by putting an \textbf{\textsc{o}} before the verb: \textbf{\textsc{o pali!}} \textit{work!}. People can be addressed by putting an \textbf{\textsc{o}} after their name: \textbf{\textsc{jan Keli o, sina pona lukin}} \textit{Kelly, you are good-looking}. When addressing people and commanding them in one sentence, one \textbf{\textsc{o}} can be dropped.
\item The word \textbf{\textsc{pi}} \textit{of} separates meanings: \textbf{\textsc{tomo telo nasa}} \textit{weird bathroom}, \textbf{\textsc{tomo pi telo nasa}} \textit{house of alcohol (bar)}. It can also be used to specifify an owner: \textbf{\textsc{tomo pi jan Lisa}} \textit{Lisa's house}.
\item We can use \textbf{\textsc{taso}} \textit{so/but/just} to join related sentences together.
\item Colors are given as mixtures/shades of the five basic colors \textbf{\textsc{jelo}} \textit{yellow}, \textbf{\textsc{laso}} \textit{blue}, \textbf{\textsc{loje}} \textit{red}, \textbf{\textsc{pimeja}} \textit{black} and \textbf{\textsc{walo}} \textit{white}.
\item There is no real number system in Toki Pona, since it can almost always be avoided to use concrete, bigger numbers, and because the fun in the language is in the simplification of our lives. That being said, it knows four numbers which can be combined to form a few larger ones: \textbf{\textsc{ala}} \textit{zero}, \textbf{\textsc{wan}} \textit{one}, \textbf{\textsc{tu}} \textit{two} and \textbf{\textsc{luka}} \textit{five}. These words can be added together to make larger numbers (\textbf{\textsc{luka tu wan}} \textit{eight}), but it is advised to use \textbf{\textsc{mute}} \textit{many} for everything else.
\item Tenses can be expressed by \textbf{\textsc{tenpo pini la ...}} (past), \textbf{\textsc{tenpo ni la ...}} (present) and \textbf{\textsc{tenpo kama la ...}} (future), if necessary.\\
\end{itemize}
\end{multicols}

\begin{center}
\begin{tabular}{cccccccc}
\toprule
a & akesi & ala & alasa & ale & anpa & ante & anu\\
(emphasis) & lizard & no & hunt & all & low & different & or\\
\midrule
awen & e & en & esun & ijo & ike & ilo & insa\\
keep & (object) & (and) & shop & thing & bad & tool & inside\\
\midrule
jaki & jan & jelo & jo & kala & kalama & kama & kasi\\
dirty & person & yellow & have & fish & sound & come & plant\\
\midrule
ken & kepeken & kili & kiwen & ko & kon & kule & kulupu\\
can & use & fruit & rock & paste & air & color & group\\
\midrule
kute & la & lape & laso & lawa & len & lete & li\\
hear & (context) & sleep & green & head & cloth & cold & (predicate)\\
\midrule
lili & linja & lipu & loje & lon & luka & lukin & lupa\\
small & line & paper & red & at & hand & see & hole\\
\midrule
ma & mama & mani & meli & mi & mije & moku & moli\\
land & parent & money & woman & me & man & eat & dead\\
\midrule
monsi & mu & mun & musi & mute & nanpa & nasa & nasin\\
back & (meow) & moon & play & many & number & strange & way\\
\midrule
nena & ni & nimi & noka & o & olin & ona & open\\
mountain & this & name & foot & (command) & love & it & open\\

\midrule
pakala & pali & palisa & pan & pana & pi & pilin & pimeja\\
break & do & stick & food & give & of & feel & black\\
\midrule
pini & pipi & poka & poki & pona & pu & sama & seli\\
end & bug & near & container & good & book & same & fine\\

\midrule
selo & seme & sewi & sijelo & sike & sin & sina & sinpin\\
skin & what & high & form & circle & new & you & face\\

\midrule
sitelen & sona & soweli & suli & suno & supa & suwi & tan\\
picture & know & animal & big & sun & table & sweet & from\\

\midrule
taso & tawa & telo & tenpo & & & & \\
but & to & water & time & & & & \\
\bottomrule
\end{tabular}
\end{center}
\end{document}
